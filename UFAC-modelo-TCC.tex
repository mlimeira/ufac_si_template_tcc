%%==============================================================
%% Adaptação do Modelo de TCC para o curso de Bacharelado em
%% Sistemas de Informação da Universidade Federal do Acre
%% Autor: Manoel Limeira (limeira@ufac.br)
%% Última versão Setembro/2019
%% Arquivo em formato UTF-8
%% Compilar com pdftex
%% Precisa do arquivo abntex2-UFAC.sty
%%==============================================================

\documentclass[
	% -- opções da classe memoir --
	12pt,				    % tamanho da fonte
	openright,			    % capítulos começam em pág ímpar (insere página vazia caso preciso)
	oneside,			    % para impressão só no anverso. Oposto a twoside
	a4paper,			    % tamanho do papel.
    % -- opções do pacote abntex2 --
    % chapter=TITLE,        % Títulos em maiúsculas
    sumario=tradicional,    % Sumário padrão memoir 
    % -- opções do pacote babel --
	english,			    % idioma adicional para hifenização
	brazil,				    % idioma principal do documento
	]{abntex2}              



% Pacotes fundamentais
\usepackage{abntex2-UFAC}        % Personalização para a Universidade Federal do Acre
\usepackage{lmodern}			% Usa a fonte Latin Modern			
\usepackage[T1]{fontenc}		% Selecao de codigos de fonte de saída
\usepackage[utf8]{inputenc}		% Codificacao do documento (conversão automática dos acentos)
\usepackage{indentfirst}		% Indenta o primeiro parágrafo de cada seção.
\usepackage{graphicx}			% Inclusão de gráficos
\usepackage{booktabs}           % \toprule, \midrule e \bottomrule para tabelas
% Sistema autor-data com títulos nas referências em negrito
\usepackage[alf,abnt-emphasize=bf]{abntex2cite}	
%Títulos em fonte Times
\usepackage{mathptmx}
\renewcommand{\ABNTEXchapterfont}{\rmfamily\bfseries}
%figuras
\usepackage[outdir=./]{epstopdf}
\usepackage{epsfig,graphicx}
%


% ---
% CONFIGURAÇÕES DE PACOTES
% ---

% Informações de dados para CAPA e FOLHA DE ROSTO
\titulo{Moledo para Trabalhos de Conclusão do Curso de Bacharelado em Sistemas de Informação da UFAC}
\autor{Nome Completo do(a) Autor(a)}
\local{Rio Branco}
\data{2019}
\orientador{Nome Completo do(a) Orientador(a)}    % redefinido no abntex2-UFAC para aceitar Instituição (default = UFAC)
%\coorientador{Nome Completo do(a) Coorientador(a)}
\instituicao{Universidade Federal do Acre}

\campus{Centro de Ciências Exatas e Tecnológicas}      % pacote abntex2-UFAC
\curso{Curso de Bacharelado em Sistemas de Informação} % pacote abntex2-UFAC
\membrobancaA{Membro(a) da Banca A}          			 % pacote abntex2-UFAC default = UFAC
\membrobancaB[UFAM]{Membro(a) da Banca B}				 % pacote abntex2-UFAC default = UFAC
\databanca{\today}                          			 % pacote abntex2-UFAC

% O preambulo deve conter o tipo do trabalho, o objetivo,
% o nome da instituição e a área de concentração
\preambulo{Trabalho de Conclusão de Curso apresentado como exigência parcial para obtenção do grau de bacharel em Sistemas de Informação da Universidade Federal do Acre.}
% ---

% ---
% Configurações de aparência do PDF final

% informações para o arquivo pdf de saída
% Interessante alterar a cor dos links para preto(black)
% na versão final para imprimir
\makeatletter
\hypersetup{
        % metadados
		pdftitle={\@title},
		pdfauthor={\@author},
    	pdfsubject={\imprimirpreambulo},
	    pdfcreator={LaTeX with abnTeX2},
		colorlinks=false,   % false: links em frame; true: links coloridos
    	linkcolor=black,   % cor dos links no documento
    	citecolor=black,    % cor dos links para a bibliografia
    	filecolor=magenta, % cor dos links para arquivos
		urlcolor=blue,     % cor dos links para sites
		bookmarksdepth=4   % profundidade do sumário do PDF
}
\makeatother
% ---

\begin{document}
% Retira espaço extra obsoleto entre as frases.
\frenchspacing

% ----------------------------------------------------------
% ELEMENTOS PRÉ-TEXTUAIS
% ----------------------------------------------------------
\pretextual

% Capa
\imprimircapa

% Folha de rosto
\imprimirfolhaderosto
% ---

% Inserir folha de aprovação
\imprimirfolhadeaprovacao

% Dedicatória
\begin{dedicatoria}
   \vspace*{\fill}
   \centering
   \noindent
   \textit{Dedicat\'oria(s): Elemento opcional, texto em que o autor presta homenagem ou dedica seu trabalho \cite{NBR14724:2011}.}
   \vspace*{\fill}
\end{dedicatoria}
% ---

% Agradecimentos
\begin{agradecimentos}
%Insira o texto de agradecimentos aqui
Elemento opcional, texto colocado ap\'os a dedicat\'oria em que o autor faz agradecimentos dirigidos àqueles que contribuíram de maneira relevante à elaboração do trabalho \cite{NBR14724:2011}. 
\end{agradecimentos}
% ---

% Epígrafe
\begin{epigrafe}
    \vspace*{\fill}
	\begin{flushright}
		\textit{``Word? nunca mais.''\\
		(Qualquer usuário de \LaTeX)}
	\end{flushright}
\end{epigrafe}
% ---

% RESUMOS

% resumo em português
\begin{resumo}
 \noindent
%Insira o resumo aqui
Segundo \cite{NBR6028:2003}, o resumo é elemento obrigat\'orio, constitu\'ido de uma sequ\^encia de frases concisas e objetivas e n\~ao de uma simples enumera\c c\~ao de t\'opicos, não ultrapassando 500 palavras. Além disso, o resumo deve ressaltar o objetivo, o método, os resultados e as conclusões do documento. A ordem e a extensão destes itens dependem do tipo de resumo (informativo ou indicativo) e do tratamento que cada item recebe no documento original. 
 \vspace{\onelineskip}

 \noindent
 \textbf{Palavras-chaves}: Palavras representativas do conteúdo do trabalho, isto é, palavras-chave e/ou descritores, conforme a ABNT NBR 6028~\cite{NBR6028:2003}.
\end{resumo}

% resumo em inglês
\begin{resumo}[Abstract]
 \begin{otherlanguage*}{english}
   \noindent
   % Insira o abstract aqui

   \vspace{\onelineskip}

   \noindent
   \textbf{Key-words}: .
 \end{otherlanguage*}
\end{resumo}

% inserir lista de ilustrações
\pdfbookmark[0]{\listfigurename}{lof}
\listoffigures*
\cleardoublepage
% ---

% inserir lista de tabelas
\pdfbookmark[0]{\listtablename}{lot}
\listoftables*
\cleardoublepage
% ---


% Lista de siglas e abreviaturas (opcional)
% sintaxe: \item [sigla] Descrição da sigla

\begin{siglas}
\item[ABNT] Absurdas Normas Técnicas
\item[UFAC] Universidade Federal do Acre
\end{siglas}

% Lista de símbolos (opcional)
% sintaxe: \item [simbolo] Descrição do símbolo

%\begin{simbolos}
%\item[$\infty$ ] Infinito
%\end{simbolos}


% inserir o sumario
\pdfbookmark[0]{\contentsname}{toc}
\tableofcontents*
\cleardoublepage
% ---



% ----------------------------------------------------------
% ELEMENTOS TEXTUAIS
% ----------------------------------------------------------
\textual


%Modifique a estrutura dos capítulos e seções de acordo com a necessidade do seu trabalho
\chapter{Introdução}\label{sec:introducao} \thispagestyle{empty}
Apresentar uma visão geral do assunto que será abordado no trabalho, procurando fazer com que o leitor adquira uma compreensão inicial do que será tratado e fornecendo informações que o levem a perceber a sua importância.  

Exemplo de Figura: Ver Figura~\ref{fig:exefig}.

\begin{figure}[!ht]
\centering
\includegraphics[width=0.3\linewidth]{figuras/exefig.eps}
\caption{Exemplo de figura.}
\label{fig:exefig}
\end{figure}

Escrever bem é uma arte que exige muita técnica e dedicação. Há vários bons livros sobre como escrever uma boa dissertação ou tese. Para a escrita de textos em Ciência da Computação, Writing for Computer Science \cite{zobel2014} é uma leitura obrigatória. O livro Metodologia de Pesquisa para Ciência da Computação \cite{wazlawick2009} também merece uma boa lida. 

Alguns links interessantes para se trabalhar com a classe abn\TeX\ e \LaTeX\ em geral\footnote{E também para usar alguns comandos de citação como exemplo}:
\begin{alineas}
  \item Informações da classe Abn\TeX : \citeonline{abntex2classe}
  \item Ajustes nas citações e referências: \citeonline{abntex2cite} e \citeonline{abntex2cite-alf}
  \item Classe memoir (base do Abn\TeX\ ): \apudonline{memoir}{abntex2classe}
  \item Livros interessantes sobre \LaTeX: \cite{Dongen2012,LeslieLamport90,FrankMittelbach111,Dongen2012}
  \item Distribuição \LaTeX\ para windows: \url{http://miktex.org/}
  \item Editor \LaTeX\ gratuito: \url{http://texstudio.sourceforge.net/}
  \item Gerenciador de arquivos \texttt{.bib}: \url{http://jabref.sourceforge.net/}
  \item Gerenciador de artigos: \url{http://www.mendeley.com/}
  \item Exemplo de Tabela: Veja \autoref{tab:cronograma}
\end{alineas}

\section{Problema da Pesquisa}\label{sec:ProbPesq}

Qual a grande questão que se busca responder? Qual o problema se presente resolver? 
O problema é a mola propulsora de todo o trabalho de pesquisa. Depois de definido o tema, levanta-se uma questão para ser respondida através de uma hipótese, que será confirmada ou negada através do trabalho de pesquisa. O Problema é criado pelo próprio autor e relacionado ao tema escolhido. O autor, no caso, criará um questionamento para definir a abrangência de sua pesquisa. Não há regras para se criar um problema, mas alguns autores sugerem que ele seja expresso em forma de pergunta.
De 1 a 2 páginas

\section{Objetivos}\label{sec:objetivos}

O objetivo da pesquisa deve ser diretamente verificável ao final do trabalho. Um bom objetivo de pesquisa possivelmente irá demonstrar que alguma hipótese sendo testada é ou não verdadeira.
\section{Objetivo Geral}\label{sec:ObjGeral}

Qual o objetivo maior do trabalho?

\section{Objetivos Específicos}\label{sec:ObjEspec}

Os objetivos específicos do trabalho devem ser expressos na forma de uma condição não trivial cujo sucesso possa vir a ser verificado ao final do trabalho. Um objetivo bem expresso em geral terá verbos como “demonstrar”, “provar”, “melhorar” (de acordo com alguma métrica definida) etc.

Deve-se tomar cuidado com certos verbos que determinam objetivos cuja verificação é trivial e, portanto, inadequada. Entre eles pode-se citar “propor”, “estudar”, “apresentar” etc. Se o objetivo do trabalho é propor algo, basta que a coisa seja proposta para que o objetivo seja atingido e, portanto, essa forma é trivial e inadequada, pois a definição do objetivo não menciona a qualidade daquilo que será proposto.

Se o objetivo do trabalho é estudar algo, então ele terá sido alcançado se aquilo foi estudado, não importando se alguma nova informação foi aprendida ou não, sendo, portanto, inadequado como objetivo de pesquisa. Estudar, normalmente, é o objetivo do aluno e não do trabalho.

Se o objetivo do trabalho consiste em apresentar algo, novamente ele é trivial e inadequado. Uma simples apresentação não produz necessariamente conhecimento novo. Por exemplo, “o objetivo deste trabalho é apresentar os operadores da lógica booleana”; tal objetivo pode ser alcançado com um pequeno texto explicando os operadores conhecidos, mas, como não traz informação nova, não é um objetivo de pesquisa.

Insira os objetivos específicos no formato de tópicos.

\section{Justificativa da Pesquisa}\label{sec:Justificativa}

A proposta, o estudo e a apresentação podem ser justificáveis como objetivo de pesquisa desde que o objeto da proposta, estudo ou da apresentação seja algo original~\cite{wazlawick2009}.

A Justificativa num projeto de pesquisa, como o próprio nome indica, é o convencimento de que o trabalho de pesquisa é fundamental de ser efetivado. O tema escolhido pelo pesquisador e a Hipótese levantada são de suma importância, para a sociedade ou para alguns indivíduos, de ser comprovada. 
Deve-se tomar o cuidado, na elaboração da Justificativa, de não se tentar justificar a hipótese levantada, ou seja, tentar responder ou concluir o que vai ser buscado no trabalho de pesquisa. A Justificativa exalta a importância do tema a ser estudado, ou justifica a necessidade imperiosa de se levar a efeito tal empreendimento.
Por que será feita a pesquisa sobre este assunto? Aqui a importância do tema. Qual a importância em se pesquisar este assunto?
 

\section{Metodologia}\label{sec:Metodologia}

Como será realizado a monografia? Pesquisa bibliográfica, estudo de caso, levantamento (ver manual sobre metodologia no grupo)
Detalhar se tem instrumentos de coleta de dados (questionário, formulário, entrevista, observação), quem é a população e a amostra. Como serão tabulados os dados.
A Metodologia é a explicação minuciosa, detalhada, rigorosa e exata de toda ação desenvolvida no método (caminho) do trabalho de pesquisa. 
É a explicação do tipo de pesquisa, do instrumental utilizado (questionário, entrevista etc), do tempo previsto, da equipe de pesquisadores e da divisão do trabalho, das formas de tabulação e tratamento dos dados, enfim, de tudo aquilo que se utilizou no trabalho de pesquisa.
Se for desenvolvimento de sistemas ou de aplicações falar quais ferramentas serão utilizadas.


\section{Organização}\label{sec:Organizacao}

O que tem em cada um dos capítulos seguintes (Detalhar a estrutura de capítulos que será adotada, partindo desde o 1º capítulo da monografia até as referências).

\chapter{Referencial Teórico}\label{sec:RefTeorico}\thispagestyle{empty}

Parte teórica a monografia: qual o assunto que dá o embasamento para a monografia? Quais os principais temas que precisam ser abordados? Dividir os temas em subtítulos. 

As referências dos documentos consultados para a elaboração do Projeto é um item obrigatório. Nela normalmente constam os documentos e qualquer fonte de informações consultadas no Levantamento de Literatura. 

Podem ser feitos diversos capítulos diferentes para a fundamentação teórica, conforme a necessidade.

Geralmente de 3 a 5 páginas baseadas em autores (de preferência livros) usando as normas metodológicas para citá-los. Referências de Internet devem ser realizadas com cuidado, não exceder a 30\% das obras citadas.

Exemplo de refer\^encias~\cite{tese2019, confinter2019, confnac2019, site2019}.

Exemplo de Tabela: ver Tabela~\ref{tab:exetab}.

\begin{table}[!ht]
\begin{center}
\caption{Distribuição IMC em Adultos}
\label{tab:exetab}
\begin{tabular}{|c |c |}
\hline
\textbf{\textbf{Classificação}} & \textbf{IMC}\\
\hline\hline
Baixo Peso & < 18,5 \\
Peso Adequado & > 18,5 e < 25 \\
Sobrepeso & > 25 e < 30 \\
Obesidade & > 30 \\
\hline
\end{tabular}
\end{center}
\end{table}

\section{Trabalhos Relacionados}\label{sec:TrabRel}

\chapter{Métodos}\label{sec:metodos}\thispagestyle{empty}
%\begin{figure}[htbp]
%  \begin{center}
%  \includegraphics[width=.5\linewidth]{LogoUFV.png}\\
%  \end{center}
%  \caption[Exemplo de Figura]{Exemplo de inserção de figura no \LaTeX. A legenda deve vir abaixo da figura. Pode usar o comando \texttt{\textbackslash legend} ou \texttt{\textbackslash fonte} para inserir a fonte da figura. Observe que na lista de ilustrações foi utilizado o nome curto fornecido como parâmetro do caption da figura (veja o arquivo fonte .tex) ao invés dessa legenda estupidamente extensa feita de forma proposital}
%  \label{fig:logo}
%  \legend{Fonte: Próprio Autor}
%\end{figure}

\chapter{Resultados}\label{sec:resultados}\thispagestyle{empty}

\chapter{Cronograma}\label{sec:cronograma}\thispagestyle{empty}
O abn\TeX\ introduziu o comando \texttt{IBGEtab} para formatação de tabelas. Um exemplo de tabela convencional do \LaTeX\ pode ser observado na \autoref{tab:cronograma} enquanto um exemplo usando o \texttt{IBGEtab} é mostrado na Tabela \ref{tab:cronogramaIBGE}.

\begin{table}[htbp]
  \centering
    \caption[Cronograma Normal]{Cronograma do Projeto em Meses}
    \label{tab:cronograma}
    \begin{tabular}{lcccccccccccc} %|c|c|c|c|c|c|c|c|c|c|c|c
    \toprule
    \textbf{Atividade} & \textbf{1} & \textbf{2} & \textbf{3} & \textbf{4} & \textbf{5} & \textbf{6} & \textbf{7} & \textbf{8} & \textbf{9} & \textbf{10} & \textbf{11} & \textbf{12} \\
    \midrule
        Revisão Bibliográfica & $\bullet$ & $\bullet$ & & & & & & & & & & \\
        Métodos & & & $\bullet$ & $\bullet$ & & & & & & & & \\
        Testes & & & & $\bullet$ & $\bullet$ & $\bullet$ & & & & & & \\
        Resultados & & & & & & & $\bullet$ & $\bullet$ & & & & \\
        Conclusão & & & & & & & $\bullet$ & $\bullet$ & $\bullet$ & & & \\
        Banca & & & & & & &&&& $\bullet$ & $\bullet$ & $\bullet$ \\
    \bottomrule
    \end{tabular}%
    \fonte{Próprio Autor}
\end{table}%



\begin{table}[htbp]
    \IBGEtab{
    \caption[Cronograma (IBGE)]{Cronograma do Projeto em Meses usando o comando IBGEtab para a formatação da tabela}
    \label{tab:cronogramaIBGE}
    }{
    \begin{tabular}{lcccccccccccc} %|c|c|c|c|c|c|c|c|c|c|c|c
    \toprule
    \textbf{Atividade} & \textbf{1} & \textbf{2} & \textbf{3} & \textbf{4} & \textbf{5} & \textbf{6} & \textbf{7} & \textbf{8} & \textbf{9} & \textbf{10} & \textbf{11} & \textbf{12} \\
    \midrule
        Revisão Bibliográfica & $\bullet$ & $\bullet$ & & & & & & & & & & \\
        Métodos & & & $\bullet$ & $\bullet$ & & & & & & & & \\
        Testes & & & & $\bullet$ & $\bullet$ & $\bullet$ & & & & & & \\
        Resultados & & & & & & & $\bullet$ & $\bullet$ & & & & \\
        Conclusão & & & & & & & $\bullet$ & $\bullet$ & $\bullet$ & & & \\
        Banca & & & & & & &&&& $\bullet$ & $\bullet$ & $\bullet$ \\
    \bottomrule
    \end{tabular}%
    }{
    \fonte{Próprio Autor}}
\end{table}%



% ----------------------------------------------------------
% ELEMENTOS PÓS-TEXTUAIS
% ----------------------------------------------------------
\postextual

% Referências bibliográficas

\bibliography{referencias}\thispagestyle{empty}

% Caso sejam necessários apêndices ou anexos em seu documento
% Use os ambientes abaixo

%% Apêndices
%
%% Inicia os apêndices
\begin{apendicesenv}
%
%% Imprime uma página indicando o início dos apêndices
\partapendices
%
\chapter{Primeiro Apêndice}
%
\chapter{Segundo Apêndice}
%
\end{apendicesenv}
%
%
%% ----------------------------------------------------------
%% Anexos
%% ----------------------------------------------------------
\begin{anexosenv}
%
%% Imprime uma página indicando o início dos anexos
\partanexos
%
\chapter{Primeiro Anexo}
%\lipsum[30]
%
\chapter{Segundo Anexo}
%\lipsum[31]
%
\end{anexosenv}

\end{document}
